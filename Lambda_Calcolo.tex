\documentclass[a4paper,11pt]{article}
\usepackage{amsmath}
\usepackage{amsthm}
\usepackage[english]{babel}
\usepackage{cleveref}
\usepackage{listing}
\usepackage{csquotes}
\usepackage[
backend=biber,
style=numeric,
sorting=ynt
]{biblatex}
\addbibresource{biblio.bib}
%environment 
\newtheorem{thm}{Theorem}
\newtheorem{lemma}{Lemma}
\newtheorem{prop}{Proposition}
\theoremstyle{definition}
\newtheorem{defn}{Definition}
\newtheorem{notation}{Notation}
\newtheorem{ese}{Example}
\newtheorem{oss}{Observation}

\newcommand{\FF}{\mathcal{F}}
\newcommand{\BB}{\mathcal{B}}
%%%
\title{An implementable definition of $\alpha$-equivalence in
$\lambda$-calculus\\
{\large Functional Programming Project}}
\author{Fabio Brau}

\begin{document}
\maketitle
\tableofcontents
\section{Introduction}


This project's aim is to provide a formal definition of $\alpha$-equivalence
on the $\lambda$-terms without seeing it as a transitive closure of a
relation and without assuming conventions on the name of the variables.


\section{Lambda Terms}
From an intuitive point of view, a $\lambda$-term is a representation of
a mathematical function written as a combination of \textit{variables}. It
is quite surprising to notice that ``a systematic notation for functions is lacking
in ordinary mathematics'' \cite{Curry}. In fact, the meaning of $f(x)$, that is the usual
accepted notation to indicate a function (\textit{Euler Notation}), is not uniquely
determined and has to be deduced from the context. Sometimes, with this
notation, we refer to a function depending on the variable $x$; sometimes we refers to the
evaluation of the function in a value (or in a point, a vector, a matrix, a
set and so on) equal to $x$.

An unambiguous way to indicate that the function $f$ depends on the variable $x$ 
could be the notation: $x\mapsto f(x)$. And, we can use the notation $f(x)$
when we want to evaluate $f$ in the object $x$. The introduction of the
symbol $\lambda$ could be helpful, in a typographic sense, by shortening the
notation $x\mapsto f(x)$ in $\lambda x.f(x)$.

\begin{defn}
  Let be $V=\{v_1,v_2,\dots\}$ an infinite set of \textit{variables}. The
  set of the $\lambda$-terms, indicated with $\Lambda$, is recursively
  defined as follow
  \begin{itemize}
    \item $V\subseteq\Lambda$, i.e., each variable is a $\lambda$-term;
    \item If $M$ is a $\lambda$-term, then $\lambda x.M\in\Lambda$, for any variable
      $x$, is a $\lambda$-term called \textit{abstraction};
    \item If $M,N$ are $\lambda$-terms then $(MN)$ is a new $\lambda$-term
      called \textit{application}.
  \end{itemize}
\end{defn}

It's compulsory to observe that the symbols $M,N,\dots, v_0, v_1,\dots$ have
to be intended as names of $\lambda$-terms. The \textit{assignment} of a name
to a $\lambda$-term, indicated with $=$, only holds and has sense in the metalanguage. 
For example, in the formula $M=\lambda x.x$, the symbol $M$ is just the label 
of the $\lambda$-term $\lambda x.x$.

By assuming that the abstraction and the application are associative on the
left (as operators of $\lambda$-terms in the metalanguage) we can introduce
the following notation.
\begin{notation}
  \label{not:left}
  Let $x,y$ be two variables and $M$ a $\lambda$-term, then the formula
  $\lambda xy.M$ is a shorter notation for $\lambda x.\lambda y. M$ that, by
  the left associative convention, represents uniquely $\lambda x.(\lambda y.M)$.
  Similarly, if $X,Y,Z$ are $\lambda$-terms, then the formula $XYZ$ uniquely
  represents $((XY)Z)$ by the left associative rule.
\end{notation}

\begin{defn}
  We will say that two $\lambda$-terms $M,N$ are \textit{syntactically
  equivalents}, and we will write $M\equiv N$, if each one can be mutually
  translated into the other trough the notation \Cref{not:left}.
\end{defn}
\subsection{Bound and Free Variables}
Let us consider the $\lambda$-expression $M=y(\lambda x.x)$. It is evident that
the two variable $x,y$ has a different meaning in the $\lambda$-term. The variable $x$, that 
appears in $M$ under the scope $\lambda$, is \textit{bound} and is not a constant of the 
$\lambda$-term. In a certain sense, that it will be more clear later, the
variable $x$ can be substituted with another variable without that $M$ loses
her meaning. Differently the variable $y$ is \textit{free} and stores a different information that
characterizes the term $M$. Observe, for example, that the term $M$ could represent the
formula $\int_0^y x\,dx$, in which the variable $x$ is usually called ``silent''.


\begin{defn}
  Given a $\lambda$-term $M$, the set $\FF(M)$ of the \textit{free variables}
  of $M$ is recursively defined as follow:
  \begin{itemize}
    \item If $M=x$, where $x\in V$, then $\FF(M)=\{x\}$;
    \item If $M=\lambda x.M_1$, then $\FF(M)=\FF(M_1)\setminus\{x\}$;
    \item If $M=M_1M_2$, then $\FF(M)=\FF(M_1)\cup\FF(M_2)$.
  \end{itemize}
  Analogously, the set $\BB(M)$ of the \textit{bound variables} of $M$
  is recursively defined as follow:
  \begin{itemize}
    \item If $M=x$, where $x\in V$, then $\BB(M)=\emptyset$;
    \item If $M=\lambda x.M_1$, then $\BB(M) = \BB(M_1)\cup \{x\}$;
    \item If $M=M_1M_2$, then $\BB(M)=\BB(M_1)\cup\BB(M_2)$;
  \end{itemize}
\end{defn}
By induction on the construction of $\Lambda$, it is easy to prove that
the definition above is well-placed.

\begin{oss}
  Observe that, by the definition above, a variable $x$ can be either a free and 
  a bound variable of a $\lambda$-term $M$. For example, the 
  $\lambda$-term $M=x(\lambda x.xx)$ is such that $\FF(M) = \BB(M) = \{x\}$.
\end{oss}

\subsection{Substitution without Capture}
Let us consider a $\lambda$-term $M$ and $x$ a free variable of $M$. In this
section we want to define the \textit{substitution operation} in order to
substitute the variable $x$ with another $\lambda$-term $N$ in $M$.

The following definition was first proposed by \cite{Curry} in order to avoid the
issue known as \textit{binding of a free variable}.

\begin{defn}[Substitution]
  \label{def:subs}
  Let $M,N\in\Lambda$, let us define the operation of \textit{substitution}
  that acts by substituting a variable $x$ in $M$ with $N$, returning a new 
  $\lambda$-term $M[x:=N]$. The
  definition is recursive on the construction of $\Lambda$.
  \begin{itemize}
    \item[{Case 1}] If $M$ is a variable:
      \begin{itemize}
        \item If $M=x$, then $M[x:=N]\equiv N$;
        \item If $M=y$ and $y\ne x$, then $M[x:=N]\equiv y$;
      \end{itemize}
    \item[Case 2] If $M=M_1M_2$ is an application:
      \begin{itemize}
        \item $M[x:=N]\equiv (M_1[x:=N])(M_2[x:=N])$; 
      \end{itemize}
    \item[Case 3] If $M$ is an abstraction:
      \begin{itemize}
        \item If $M=\lambda x.M_1$, then $M[x:=N] \equiv M$;
        \item If $M =\lambda y.M_1$ and $x\ne y$, then
          \[
            M[x:=N] \equiv\lambda z.M_1[y:=z][x:=N]
          \]
          where: $z=y$ if $x\not\in\FF(M_1)$ or $y\not\in\FF(N)$; 
          $z$ is choose to be not in $M$ and not in $N$ otherwise;
      \end{itemize}
  \end{itemize}
\end{defn}

The first and the second case are intuitive. Regarding the third case few
more words are required. If $M$ is an abstraction of the form $\lambda
x.M_1$, substituting $x$ with $N$ has no meaning because $x$ is bound.
Differently, if we want to substitute $x$ in a $\lambda$-term of the form
$\lambda y.M_1$ (and $x\ne y)$, then consider the following example. Let
$M=\lambda y.x$ and $N=y$. Observe that $y$ is free in $N$ and is bound in $M$,
and if we change $x$ with $N$ without introducing a new variable $z$ we will
obtain $\lambda y.y$ that is not equivalent, in a sense that it will be more
clear later, to the desired result. This phenomena is called \textit{binding}
of a free variable.

\begin{lemma}
  For any $M,N\in\Lambda$, the following statements hold.
  \begin{enumerate}
    \item If $x\not\in\FF(M)$ and $x\not\in\BB(M)$, then $M[x:=N]\equiv M$;
    \item If $y\not\in\FF(M)$ and $y\not\in\BB(M)$, then $M\equiv M[x:=y][y:=x]$;
    \item If $x\in\FF(M)$ and $y\not\in\FF(M)$, then $x\not\in\FF(M[x:=y])$ e
      $y\in\FF(M[x:=y])$;
  \end{enumerate}
  \label{lem:sost}
  \begin{proof}
    Each statement can be proved by (strong) induction on the construction of the
    $\lambda$-terms. The key idea is to observe that $\Lambda =
    \cup_{n=0}^\infty\Lambda_n$ where: $\Lambda_0=V$ and $\Lambda_n$ contains all the
    $\lambda$-terms that can be generated by application or abstraction of $\lambda$-terms 
    in $\Lambda_k$ with $k<n$. After that, by supposing that independently $1,2,3$ hold
    for $\Lambda_k$ with $k<n$, it is easy by applying the \Cref{def:subs}
    to prove that the statement hold also for $\Lambda_n$.
  \end{proof}
\end{lemma}

\section{$\alpha$-equivalence}
From an intuitive point of view two $\lambda$-terms are $\alpha$-equivalents
if each one can be transformed in the other by changing one or more bound
variables. This definition hides a number of complications, and it is not
easy to state it in a formal manner. 

Church, \cite{Church}, solves the problem by including the $\alpha$-equivalence in a semantic
level by adding the statement $\lambda x.M = \lambda z.M[x:=z]$ (called
$\alpha$) in the list of the \textit{conversion rules}. Moreover, 
he required different assumptions on the construction of the
$\lambda$-terms, for example in his formulation only abstraction trough free
variables was accepted ($\lambda y.x$ is not considered a $\lambda$-term).

In Curry's formulation \cite{Curry} a different idea is used: 
\begin{enumerate}
  \item The definition of $\alpha$-equivalence is provided for abstractions
    by stating $\lambda x.M \equiv_\alpha \lambda z.M[x:=z]$. 
  \item The relation $\equiv_\alpha$ is then extended to the whole
$\lambda$-terms by transitive closure. 
\end{enumerate}

In \cite{Barendregt}, the definition of $\alpha$-equivalence is included in the
syntactic equivalence and it is defined informally as: ``M is
$\alpha$-congruent to $N$, if $N$ results from $M$ by a series of changes of
the bound variables''. Moreover, the author needs a variable convention: 
When two terms are in the same statement (or definition,
theorem, etc.) then different names for the variables have to be used.

Even if the definition above are totally valid, by a practical point of view
they contains two uncomfortable things: They don't provide an
operative definition \cite{Curry}. They use conventions on the terms
(\cite{Church, Barendregt}). The following lines provide an operative definition of
$\alpha$-equivalence that doesn't requires further conventions or assumption
and that can be easily implemented trough a functional paradigm.

\begin{defn}[$\alpha$-equivalence]
  \label{def:alpha_eq}
  Let us define $\equiv_\alpha$ recursively and in parallel with the
  construction of $\Lambda$.
  Let $\Lambda_0$ be the set of the terms of rank $0$ (i.e, of the form $x$ for
  each variable in $V$), we define
  \[
    \forall x,y \in \Lambda_0,\quad x\equiv_\alpha y\iff x=y.
  \]
  Let us consider the set $\Lambda_k$ of the $\lambda$-terms of rank $k$
  defined as $\Lambda_k = \hat\Lambda_k\cup \bar\Lambda_k$ where
  \[
    \begin{aligned}
      \hat \Lambda_k &=\left\{ \lambda
      x.M\,:M\in\Lambda_{k-1},\,x\in V\right\}\\
      \bar \Lambda_k&=\left\{ (MN), (NM)\,:\, M\in\Lambda_{k-1},\,N\in\Lambda_i,\,i<k \right\}
    \end{aligned}
  \]
  Let us define over this set the relation $\equiv_\alpha$ as follow
  \begin{itemize}
    \item If $M,N\in\hat\Lambda_{k}$, with $M=\lambda x.M_1$ and $N=\lambda
      y.N_1$, then
      \begin{itemize}
        \item If $x=y$, then
          \[
            M\equiv_\alpha N \iff M_1\equiv_\alpha N_1
          \]
        \item If $x\ne y$, then
          \[
            M\equiv_\alpha N \iff M_1\equiv_\alpha N_1[y:=x]\quad\wedge\quad
            (y\not\in\FF(M_1) \wedge x\not\in\FF(N_1))
          \]
      \end{itemize}
    \item If $M,N\in\bar\Lambda_k$, with $M=M_1M_2$ and $N=N_1N_2$, then
      \[
        M\equiv_\alpha N \iff M_1\equiv_\alpha N_1\wedge M_2\equiv_\alpha N_2
      \]
  \end{itemize}

  Observing that $\Lambda = \cup_{k=0}^\infty\Lambda_k$, and observing that
  $\Lambda_i\cap\Lambda_j=\emptyset$, we can extend $\equiv_\alpha$ to the
  whole set of $\lambda$-terms.
\end{defn}

\begin{prop}
  The relation $\equiv_\alpha \subseteq \Lambda\times\Lambda$ is an
  equivalence.
  \begin{proof}
    We want to proceed by strong induction over the rank.
    By construction $\equiv_\alpha$ is an equivalence over $\Lambda_0$. Let
    us suppose that $\equiv_\alpha$ is an equivalence for $\Lambda_i$ with
    $i<k$. We want to prove \textit{reflexivity, symmetry and transitivity} property.
    \begin{itemize}
      \item[Reflex.] If $M\in\hat\Lambda_k$ with $M=\lambda x.M_1$, then
        $M\equiv_\alpha M\iff M_1\equiv_\alpha M_1$ that is true because
        $\equiv_\alpha$ is of equivalence on $\Lambda_{k-1}$. If
        $M\in\bar\Lambda$ with $M=M_1M_2$, then $M\equiv_\alpha M$ if and only 
        if $M_1\equiv_\alpha M_1$ and $M_2\equiv_\alpha M_2$, that is true 
        because $\equiv_\alpha$ is of equivalence on $\Lambda_i$ with $i<k$. 
      \item[Symmet.]
        Let $M\equiv_\alpha N$. If $M,N\in\hat\Lambda_k$ with $M=\lambda x.M_1$
        and $N=\lambda y.N_1$, then by construction: if $x=y$, $M_1\equiv_\alpha
        N_1$ implies that $N_1\equiv_\alpha M_1$; otherwise, if $x\ne y$,
        then $M_1\equiv_\alpha N_1[y:=x]$ (with $x\not\in\FF(N_1)$ and
        $y\not\in\FF(M_1)$) and so by \Cref{lem:sost} $M_1[x:=y]\equiv_\alpha 
        N_1[y:=x][x:=y]\equiv N_1$. The case $M,N\in\bar\Lambda_k$ follows
        from Inductive Hypothesis.
      \item [Trans.] Suppose that $X,Y,Z\in\Lambda_k$, $X \equiv_\alpha Y$,
        and $Y\equiv_\alpha Z$. If $X,Y,Z\in\hat\Lambda_k$ with 
        $X=\lambda x.X_1$, $Y=\lambda y.Y_1$, $Z=\lambda z.Z_1$ then for
        point 3 of \Cref{lem:sost} $x\not\in\FF(Z_1)$ and so
        $X_1\equiv_\alpha Y_1[y:=x]\equiv_\alpha Z_1[z:=y][y:=x]\equiv_\alpha
        Z_1[z:=x]$. Therefore, if $X,Y,Z\in\bar\Lambda_k$, then, because 
        the transitive property is valid for the single terms of the
        applications in $\bar\Lambda_i$ with $i<k$, then by inductive
        hypothesis the property also hold for $\Lambda_k$.
    \end{itemize}
  \end{proof}
\end{prop}

\section{Implementation}
All the definitions in the previous sections can be implemented in Huskell
language. 
\subsection{$\lambda$-terms}
The $\lambda$-terms are formally defined as a data type as follow
\begin{verbatim}
-- Definition of lambda terms
data LamTerm = Var String | Abs String LamTerm | App LamTerm LamTerm
\end{verbatim}
The following function checks whether a variable $x$ is free in a term $M$.
The definition is made by cases on the construction of the type
\texttt{LamTerm}.
\begin{verbatim}
-- Function that check if a variable is free
is_free x (Var y)     = (x == y)
is_free x (Abs y e1)  = if (x==y) then False else is_free x e1
is_free x (App e1 e2) = is_free x e1 || is_free x e2 
\end{verbatim}

In a similar manner we can define the function \texttt{is\_bound} that check
whether a variable is bound in a $\lambda$-term
\begin{verbatim}
-- Function that check if a variable is bound
is_bound x (Var y)     = False
is_bound x (Abs y e1)  = if (x==y) then True else is_bound x e1
is_bound x (App e1 e2) = is_bound x e1 || is_bound x e2 
\end{verbatim}

\subsection{Substitution}
Before presenting the \texttt{subs} function, we have to notice that when a
$\lambda$-term is an abstraction (Case 3 of \Cref{def:subs}) we carefully
need to introduce a new fresh variable. The following functions allows to
provide a new variable that is not present in a term neither as free or bound
variable.

\begin{verbatim}
-- Function that generate fresh Var
-- Auxiliary function that generate a new variable from an old name
fresh_from_str m y = 
  if not ((is_free y m) || (is_bound y m)) then 
    y 
  else
    fresh_from_str m (y ++ "+")

-- fresh m --> String, return a string the is not free and not bound in m 
fresh m = fresh_from_str m "a"
\end{verbatim}

Finally, the function \texttt{subs} take respectively a variable $x$, a term
$M$ and a term $N$ and provides the $\lambda$-term $M[x:=N]$ by using \Cref{def:subs}.

\begin{verbatim}
-- Substitution function:
-- (subs x m n) substitutes the variable x in m with the lambda term n
--
-- Case 1
subs x (Var y) n = if (x == y) then n else (Var y)

-- Case 2
subs x (App m1 m2) n = App (subs x m1 n) (subs x m2 n)

-- Case 3
subs x (Abs y m) n = 
  if (x == y) then 
    (Abs y m) 
  else 
    if not ((is_free x m) && (is_free y n)) then 
      Abs y (subs x m n)
    else 
      Abs (fresh (App m n)) (subs x (subs y m (Var (fresh (App m n)))) n)

\end{verbatim}

\subsection{\alpha-equivalence}

In conclusion, the function \texttt{alpha\_eq} checks whether two $\lambda$-terms are
$\alpha$-equivalents by following the \Cref{def:alpha_eq}. 

\begin{verbatim}
-- Check for Alpha equivalence
alpha_eq (Var x) (Var y) = (x == y)
alpha_eq (App m1 m2) (App n1 n2) = (alpha_eq m1 n1) && (alpha_eq m2 n2)
alpha_eq (Abs x m) (Abs y n) = 
  if (x==y) then 
    alpha_eq m n
  else
    (not(is_free y m)) && (not(is_free x n)) && (alpha_eq m (subs y n (Var x)))
-- Spurious cases
alpha_eq _ _ = False
\end{verbatim}
Observe that the last line is compulsory in order to define the function in all
the possible cases and to avoid
\texttt{Non-exhaustive patterns in function alpha\_eq} error. 
\section{Examples}
The following lines provide a list of examples of utilizations of the
previous definitions.

It is helpful to introduce the function \texttt{l2s} that prints an object of
type \texttt{LamTerm} as as string. The function is recursively defined on
the definition of the type
\begin{verbatim}
-- Prinitng function
l2s (Var n)     = n
l2s (App e1 e2) = "(" ++ (l2s e1) ++ " " ++ (l2s e2) ++ ")"
l2s (Abs n e)   = "\\" ++ n ++ "." ++ (l2s e)
\end{verbatim}

We can put all the definition above in the module \texttt{LambdaCalcolo.hs}
and load it in \texttt{ghci} 
\begin{verbatim}
GHCi, version 8.4.4: http://www.haskell.org/ghc/  :? for help
Prelude> :load LambdaCalcolo.hs 
[1 of 1] Compiling Main             ( LambdaCalcolo.hs, interpreted )
Ok, one module loaded.
\end{verbatim}

Initially we can define and print two terms
\begin{verbatim}
*Main> m = Abs "x" (App (Var "x") (Var "y"))
*Main> 
*Main> l2s m
"\\x.(x y)"
*Main> 
*Main> 
*Main> n = Abs "z" (App (Var "z") (Var "x"))
*Main> 
*Main> l2s n
"\\z.(z x)"
\end{verbatim}

We can check the presence of bound and free variables by running
\begin{verbatim}
*Main> is_free "x" m
False
*Main> is_bound "x" m
True
\end{verbatim}

We can now check whether the substitution is made without capture by running
\begin{verbatim}
*Main> l2s (subs "y" m (Var "x"))
"\\a.(a x)"
*Main>
*Main> l2s (subs "y" m n)
"\\a.(a \\z.(z x))"
*Main>
\end{verbatim}

And finally we can observe that the two terms $m$ and $n$ are not
equivalents.
\begin{verbatim}
*Main> alpha_eq m n
False
*Main>
*Main> alpha_eq m (subs "x" n (Var "y"))
True
*Main> 
\end{verbatim}
%%%%
% BIBLIOGRAPHY
%%%%
\printbibliography

\end{document}
\section{Equivalenza Semantica}
Possiamo ora definire una prima versione della equivalenza semantica come segue
\begin{defn}[Teoria $\lambda$]
  Teoria del primo ordine sul linguaggio $\Lambda$ con relazione di equivalenza semantica 
  $\dot =$ per cui valgono i seguenti assiomi
  \begin{itemize}
    \item[($\alpha$)] $M\equiv_\alpha N \Rightarrow M \dot = N$;
    \item[($\beta$)] $(\lambda x.M)N \dot = M[x:=N]$;
    \item[$(\xi)$] $M\dot = N \Rightarrow \lambda x.M \dot = \lambda x.N$;
    \item[$(I)$] $\dot = \subseteq \Lambda\times\Lambda$ è di equivalenza;
    \item[$(II)$] $M\dot = N \Rightarrow \quad ZM\dot =ZN \wedge\, MZ\dot
      =NZ$;
  \end{itemize}
\end{defn}


